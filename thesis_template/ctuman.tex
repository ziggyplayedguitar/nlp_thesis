% arara: pdflatex: { synctex: yes }
% arara: makeindex: { style: ctuthesis }
%% arara: bibtex

%\listfiles


%\PassOptionsToPackage{cp1250}{inputenc}

% The class takes all the key=value arguments that \ctusetup does,
% and couple more: draft and oneside
\documentclass[twoside]{ctuthesis}

\makeatletter
\edef\mytoday{\expandafter\@gobbletwo\the\year\ifnum\month<10 0\fi\the\month\ifnum\day<10 0\fi\the\day}
\makeatother

% LaTeX logo with better kerning in sf bf font
\makeatletter
\newcommand\LaTeX@lmss@bx{L\kern -.33em{\sbox \z@ T\vbox to\ht \z@ {\hbox {\check@mathfonts \fontsize \sf@size \z@ \math@fontsfalse \selectfont A}\vss }}\kern -.15em\TeX}
\DeclareRobustCommand\myLaTeX{%
	\ifcsname LaTeX@\f@family @\f@series\endcsname
		\csname LaTeX@\f@family @\f@series\endcsname
	\else
		\LaTeX
	\fi
}

\ctusetup{
%	preprint = {\ctuverlog \\ ctuman \mytoday},
	mainlanguage = english,
%	titlelanguage = english,
%	otherlanguages = {czech},
	% title-czech = {Manuál ke třídě ctuthesis pro {\myLaTeX}},
	title-english = {NLP Trolls},
	doctype-english = {Bachelor thesis},
%	xfaculty = F4,
%	department-czech = {Katedra matematiky},
%	department-english = {Department of Mathematics},
	author = {Luka Peraica },
	supervisor = {Ing. Radek Mařík, CSc.},
%	supervisor-address = {Ústav X, \\ Uliční 5, \\ Praha 99},
	keywords-czech = {manuál, závěrečnná práce, \LaTeX},
	keywords-english = {manual, degree project, \LaTeX},
	day = 16,
	month = 4,
	year = 2024,
%	list-of-figures = false,
%	list-of-tables = false,
%	monochrome = true,
%	savetoner = true,
	pkg-listings = true,
	ctulstbg = none,
%	layout-short = true,
%	pkg-hyperref = false,
}

\ctuprocess

% Theorem declarations, this is the reasonable default, anybody can do what they wish.
% If you prefer theorems in italics rather than slanted, use \theoremstyle{plainit}
\theoremstyle{plain}
\newtheorem{theorem}{Theorem}[chapter]
\newtheorem{corollary}[theorem]{Corollary}
\newtheorem{lemma}[theorem]{Lemma}
\newtheorem{proposition}[theorem]{Proposition}

\theoremstyle{definition}
\newtheorem{definition}[theorem]{Definition}
\newtheorem{example}[theorem]{Example}
\newtheorem{conjecture}[theorem]{Conjecture}

\theoremstyle{note}
\newtheorem*{remark*}{Remark}
\newtheorem{remark}[theorem]{Remark}

% Marginpars used as navigation aids.
\usepackage{mparhack}

\newcommand\indexmp[1]{{\sffamily\bfseries#1}}

\ExplSyntaxOn
\cs_new:Nn \ctuman_domarginpar:n {
	\marginpar
	[ \raggedleft \footnotesize \sffamily #1 ]
	{ \raggedright \footnotesize \sffamily #1 }
}
\cs_generate_variant:Nn \ctuman_domarginpar:n { x }
\DeclareDocumentCommand \ctump { m } {
	\clist_set:Nn \ctuman_temp_clist { #1 }
	\ctuman_domarginpar:x { \clist_use:Nnnn \ctuman_temp_clist { \\ } { \\ } { \\ } }
	\clist_map_inline:Nn \ctuman_temp_clist { \index{##1|indexmp} }
	\ignorespaces
}
\ExplSyntaxOff

% Abstract in Czech
\begin{abstract-czech}
V záplavě mnoha zdrojů a množství mediálních zpráv není jednoduché se zorientovat i pro profesionální mediální analytiky. 
Výrazem demokracie je i možnost se ke zprávám vyjadřovat a tříbit si názory v diskusních příspěvcích dílčích zpráv. 
Diskuse však vytváří prostor i pro osoby, jejichž cílem je z rozmanitých důvodu diskuse narušovat a překrucovat. 
Cílem práce je vytvořit komponenty systému, který umožní sledovat linie vývoje tématu a identifikovat příspěvky narušitelů, 
tzv. trollů.\end{abstract-czech}

% Abstract in English
\begin{abstract-english}
	
\end{abstract-english}

% Acknowledgements / Podekovani
\begin{thanks}
	We thank the CTU in Prague for being a~very good \emph{alma mater}.
\end{thanks}

% Declaration / Prohlaseni
\begin{declaration}
	I declare that this work is all my own work and I have cited all sources I have
	used in the bibliography.

\medskip

	Prague, \monthinlanguage{title} \ctufield{day}, \ctufield{year}

\vspace*{2cm}

	Prohlašuji, že jsem předloženou práci vypracoval samostatně, a že jsem uvedl veškerou použitou literaturu.

\medskip

	V Praze, \ctufield{day}.~\monthinlanguage{second}~\ctufield{year}
\end{declaration}

\usepackage{url}

\usepackage{tabularx,array}

\usepackage{mathtools,amssymb}

% A savebox for typesetting listings in the titles
\newsavebox{\myboxa}

%\newcommand*\symbO{$\color{red}\bowtie$}
\newcommand*\symbO{\raisebox{0.5\height}{\scalebox{0.7}{\color{red}${\vartriangleright}\mkern-6mu{\vartriangleleft}$}}}
\newcommand*\symbM{\raisebox{0.5\height}{\scalebox{0.7}{\color{red}${\blacktriangleright}\mkern-6mu{\blacktriangleleft}$}}}
\newcommand*\itemO{\item\leavevmode\kern-0.33em\symbO}
\newcommand*\itemM{\item\leavevmode\kern-0.33em\symbM}



\begin{document}

% We actually don't want inline listings to have a background color
\renewcommand \ctulstsep{0pt}

% \ctuclsname for typesetting the class' name
\newcommand\ctuclsname{\leavevmode\unhcopy\ctuclsnamebox}
\newsavebox\ctuclsnamebox
\begin{lrbox}{\ctuclsnamebox}
\ctulst!ctuthesis!
\end{lrbox}

\maketitle

% ========================================== CHAPTER 1 INTRODUCTION ==============================
\chapter{Introduction}

\section{Problem Statement}
\par
In today’s flood of diverse media sources and information, even professional media analysts find it challenging to navigate and filter reliable content. A key aspect of democracy is the ability to express opinions and refine perspectives through discussions on news articles. However, these online discussions also create opportunities for individuals whose goal is to disrupt and manipulate conversations for various reasons. The rise of online trolling has become a significant issue, as trolls deliberately provoke, mislead, and incite conflict, thereby spreading misinformation and fostering hostility in digital spaces.\par
The internet, as a central platform for communication, information sharing, and community building, is increasingly affected by this phenomenon. Studies, such as that by Fornacciari et al.\cite{Fornacciari2018}, demonstrate that different types of trolls display unique behavioral patterns, emphasizing the need for diverse and adaptive detection methods. Natural Language Processing (NLP) has emerged as a crucial tool in addressing this challenge, offering methods to automatically identify and mitigate the impact of trolls. This thesis aims to develop components of a system capable of tracking the evolution of discussion topics and identifying disruptive contributions from trolls. It provides an overview of various NLP techniques for troll detection, including stylometry, topic modeling, deep learning, and transformer models.

\section{Structure of the Thesis}

% ========================================== CHAPTER 2 THEORETICAL BACKGROUND =====================
\chapter{Theoretical Background}
\label{chap:theory}

\section{Stylometry}
Stylometry is the discipline of analyzing writing style to uncover patterns, identify authors, and extract meaningful details from texts.\cite{Mosteller1964Federalist}\cite{Pascucci2020Misogyny} The term was introduced in 1890 by the Polish philosopher Wincenty Lutosławski, who applied it to analyze Plato's works.\cite{Lutoslawski1898} In the context of this thesis,  stylometry involves the use of automated techniques to analyze linguistic traits that distinguish authors based on their unique writing patterns.\par
One of the core assumptions in computational stylometry is that an author’s choices are influenced by sociological factors, such as age, gender, and education level, as well as psychological factors like personality and native language proficiency.\cite{Daelemans2013Explanation} These choices form a distinct, recognizable style that can be analyzed for various purposes, including troll detection. Stylistic features, which play a fundamental role in this process, range from simple surface-level metrics like word length to more complex syntactic and semantic traits.\par
We can group these features into key categories studied in literature:
\begin{itemize}
    \item \textbf{Lexical Features:} These can be word choices, vocabulary richness or usage of certain phrases.
    \item \textbf{Syntactic Features:} This involves sentence structure, punctuation usage and grammatical complexity.\cite{Sari2018Features}
    \item \textbf{Semantic Features:} Which explores meaning and sentiment express in a text.\cite{Jiang2021Sentiment}
\end{itemize}
By extracting these features, machine learning classifiers can be trained to recognize troll behavior.

\section{Topic Detection Techniques}

Topic detection methods, such as Latent Dirichlet Allocation (LDA), can be employed to analyze the thematic content of online messages and identify patterns associated with troll behavior. LDA is a probabilistic model that identifies latent topics in a collection of documents based on word co-occurrence patterns.\cite{Blei2001LDA} It assumes that each document is a mixture of topics, and each topic is a distribution over words. By analyzing the distribution of words in troll messages, LDA can uncover the underlying topics that trolls frequently discuss. However, challenges exist in selecting the appropriate algorithm and determining the optimal number of topics.\cite{Ruediger2022TopicModellingRevisited} Research has shown that LDA can be effectively used to analyze troll tweets during events like the 2016 US election, revealing coordinated campaigns focused on specific political issues.\cite{Golino2022Elections}

BERTopic is another state-of-the-art technique that can be used for dynamic topic modeling. It leverages pre-trained transformers and Class-based TF-IDF to create dense clusters allowing for easily interpretable topics while keeping important words in the topic descriptions. BERTopic also allows for the analysis of topic evolution by calculating the topic representation at different timestep without the need to run the entire model several times.\cite{Grootendorst2022BERTopic}

\section{Sentiment Analysis}


% ========================================== CHAPTER 3  ==========================================
\chapter{}



\appendix

\printindex

\bibliographystyle{plain}
\bibliography{../sources/library.bib}

\end{document}